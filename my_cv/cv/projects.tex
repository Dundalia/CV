%-------------------------------------------------------------------------------
%	SECTION TITLE
%-------------------------------------------------------------------------------
\cvsection{Projects}

\newcommand\gitscale{0.015}
\newcommand\bpscale{0.028}

%-------------------------------------------------------------------------------
%	CONTENT
%-------------------------------------------------------------------------------
\begin{cventries}


\cventry
  {Personal Open-Source Project} % Organisation
  {\href{https://github.com/Dundalia/agent-studio}{Full-Stack AI Agent Platform with LangGraph Integration}} 
  {Montreal, Canada} % Location
  {Jul 2025} % Date(s)
  {
    \begin{cvitems} % Description(s) of project
      \item {Built a comprehensive full-stack platform for creating and deploying custom AI agents (like DeepResearch) using modern web technologies. Frontend developed with Next.js, React, and TypeScript. Backend implemented with FastAPI and LangGraph framework, supporting multiple AI models.}
    \end{cvitems}
  }
%---------------------------------------------------------
\cventry
  {Creator and Lead Developer} % Organisation
  {\href{https://choosr.it/esplora-carriere}{Choosr -- Career Orientation Platform}} 
  {Remote} % Location
  {Sept 2024 -- Present} % Date(s)
  {
    \begin{cvitems} % Description(s) of project
      \item {Designed and implemented the entire technical stack for a career guidance platform helping Italian high school students navigate university choices. }
      \item {Built and integrated comprehensive databases of career paths and Italian university programs.}
      \item {Implemented the RIASEC psychometric test to provide personalized career recommendations.}
    \end{cvitems}
  }
%---------------------------------------------------------
% \cventry
%   {Personal Project} % Organisation
%   {\href{https://belleparoleaps.it}{\includegraphics[scale=\bpscale]{images/belleparole.png}} 
%   belleparoleaps.it Website Development} 
%   {Florence, Italy} % Location
%   {Sept 2024} % Date(s)
%   {
%     \begin{cvitems} % Description(s) of project
%       \item {Developed the official website for Belle Parole APS using HTML, CSS, and Webflow.}
%     \end{cvitems}
%   }


% \cventry
%   {National Institute of Informatics} % Organisation
%   {\href{https://github.com/Dundalia/TWOLAR}{\includegraphics[scale=\gitscale]{images/github.png}} 
%   TWOLAR: a TWO-step LLM-Augmented distillation method for passage Reranking} 
%   {Tokyo, Japan} % Location
%   {Jan 2024} % Date(s)
%   {
%     \begin{cvitems} % Description(s) of project
%       \item {This repository hosts the implementation for "TWOLAR", a novel method for passage reranking. The TWOLAR approach involves a two-step distillation process augmented by language models, enhancing the performance in various reranking benchmarks.}
%       \item {Key features include a comprehensive framework for training and evaluating the TWOLAR method, supporting different reranking datasets like BEIR and TREC-DL2019/2020.}
%       \item {The repository includes detailed instructions for training the model, evaluating performance, and understanding the unique approach of TWOLAR, including language model augmentation.}
%       \item {Special focus is given to replicability and ease of use, ensuring that researchers and practitioners can easily adapt the TWOLAR method for their specific needs in passage reranking challenges.}
%     \end{cvitems}
%   }


%---------------------------------------------------------
\cventry
  {Personal Open-Source Project} % Organisation
  {\href{https://github.com/Dundalia/pytorch-deepdream}{Deepdreaming with newest vision architectures}} 
  {Tokyo, Japan} % Location
  {Aug 2023} % Date(s)
  {
    \begin{cvitems} % Description(s) of project
      \item {Extended and modernized one of the most comprehensive open-source repositories for DeepDream-style visualizations, adding support for the latest vision models (CLIP, ConvNeXt, and more).}
    \end{cvitems}
  }

%---------------------------------------------------------
  % \cventry
  %   {University of Bologna} % Organisation
  %   {\href{https://github.com/Dundalia/Sparse_retrieval_models}{\includegraphics[scale=\gitscale]{images/github.png}} Comparison of \textit{bag-of-words} text retrieval models} % Project
  %   {Bologna, Italy} % Location
  %   {Mar 2023} % Date(s)
  %   {
  %     \begin{cvitems} % Description(s) of project
  %       \item {Implemented TF-IDF and several BM25 variants, and evaluated  on three different information retrieval datasets.}
  %     \end{cvitems}
  %   }

%---------------------------------------------------------
% \cventry
%     {University of Bologna} % Organisation
%     {\href{https://github.com/Dundalia/ex_ex_gen_multi-hop_QA}{\includegraphics[scale=\gitscale]{images/github.png}} A Pipeline for an explainable, extractive-generative multi-hop QA model} % Project
%     {Bologna, Italy} % Location
%     {Feb 2023} % Date(s)
%     {
%       \begin{cvitems} % Description(s) of project
%         \item {Implemented a pipeline for an extractive-generative Question-Answering model featuring a Retriever-Reader architecture, \\trained on the HotPotQA dataset.}
%       \end{cvitems}
%     }
  
%---------------------------------------------------------
% \cventry
%     {University of Bologna} % Organisation
%     {\href{https://github.com/Dundalia/Friendship_paradox}{\includegraphics[scale=\gitscale]{images/github.png}} A NetLogo simulation for the study of the friendship paradox} % Project
%     {Bologna, Italy} % Location
%     {Feb 2023} % Date(s)
%     {
%       \begin{cvitems} % Description(s) of project
%         \item {Studied the robustness of the Friendship Paradox running simulation with NetLogo.}
%       \end{cvitems}
%     }
  
%---------------------------------------------------------
% \cventry
%     {University of Bologna} % Organisation
%     {\href{https://github.com/Dundalia/POS_tagging_with_neural_architectures}{\includegraphics[scale=\gitscale]{images/github.png}} POS tagging with different recurrent neural architectures} % Project
%     {Bologna, Italy} % Location
%     {Dec 2022} % Date(s)
%     {
%       \begin{cvitems} % Description(s) of project
%         \item {Tuned and evaluated a few modifications of a baseline BiDirectional LSTM on a POS tagging corpus.}
%       \end{cvitems}
%     }
  
%---------------------------------------------------------
% \cventry
%     {University of Bologna} % Organisation
%     {\href{https://github.com/Dundalia/Conversational_QA_CoQA}{\includegraphics[scale=\gitscale]{images/github.png}} Question Answering with Encoder-Decoder architectures on CoQA} % Project
%     {Bologna, Italy} % Location
%     {Dec 2022} % Date(s)
%     {
%       \begin{cvitems} % Description(s) of project
%         \item {Implemented four different encoder-decoder models, based on Pretrained BERT-like architectures, and studied the effect of inputting the history of the dialogues.}
%       \end{cvitems}
%     }
  
%---------------------------------------------------------
% \cventry
%     {University of Bologna} % Organisation
%     {\href{https://github.com/Dundalia/LoopQPrize_2022}{\includegraphics[scale=\gitscale]{images/github.png}} Loop Q prize competition - Emotion detection from speech} % Project
%     {Bologna, Italy} % Location
%     {May 2022} % Date(s)
%     {
%       \begin{cvitems} % Description(s) of project
%         \item {Implemented a solution based on multi audio features extraction and a BiLSTM architecture to tackle the emotion recognition task on the datasets: CREMA, SAVEE, TESS, RAVDESS. }
%       \end{cvitems}
%     }
  
%---------------------------------------------------------
% \cventry
%     {University of Bologna} % Organisation
%     {\href{https://github.com/drchapman-17/cdmo2022}{\includegraphics[scale=\gitscale]{images/github.png}} Combinatorial optimization solution to VLSI} % Project
%     {Bologna, Italy} % Location
%     {Jul 2022} % Date(s)
%     {
%       \begin{cvitems} % Description(s) of project
%         \item {Implemented a combinatorial optimization approach to the Very Large Scale Integration problem with four different technologies: Constraint Programming, Propositional SATisfiability, Satisfiability Modulo Theories, and Linear Programming. }
%       \end{cvitems}
%     }
  
%---------------------------------------------------------

% \cventry
%     {University of Bologna} % Organisation
%     {\href{https://github.com/Simone999/AshtonTablutAgent}{\includegraphics[scale=\gitscale]{images/github.png}} Intelligent agent that plays Tablut} % Project
%     {Bologna, Italy} % Location
%     {Dec 2021} % Date(s)
%     {
%       \begin{cvitems} % Description(s) of project
%         \item {Developed an intelligent system based on the Monte Carlo Tree Search algorithm able to play Tablut, an ancient board game.}
%       \end{cvitems}
%     }
  
%---------------------------------------------------------


\end{cventries}
