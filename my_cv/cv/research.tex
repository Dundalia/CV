%-------------------------------------------------------------------------------
%	SECTION TITLE
%-------------------------------------------------------------------------------
\cvsection{Research Experience}


%-------------------------------------------------------------------------------
%	CONTENT
%-------------------------------------------------------------------------------
\begin{cventries}
%---------------------------------------------------------

  \cventry
    {\href{https://amslaurea.unibo.it/30082/}{MSc Thesis} - International Internship Program} % Job title
    {National Institute of Informatics} % Organisation
    {Tokyo, Japan} % Location
    {Mar 2023 - Aug 2023} % Date(s)
    {
      \begin{cvitems} % Description(s) of tasks/responsibilities
        \item {Supervisors: Akiko Aizawa (National Institute of Informatics), Paolo Torroni (University of Bologna)} 
        \item {Title: A TWO-step LLM Augmented distillation method for passage Reranking}
        % \item {Conducted research on Text Information Retrieval. We have investigated recent advancements in open-sourced Large Language Models, text retrievers, rerankers, knowledge distillation and ranking losses. Our approach is to adapt a T5 model to solve the reranking task, training it on GPT augmented labels. We have obtained SoTA results on the TREC-DL test sets and on the BEIR benchmark.}
      \end{cvitems}
    }

%---------------------------------------------------------
  \cventry
    {\href{https://github.com/Freddavide/Freddavide/blob/main/elaborato_B036_Baldelli_Davide.pdf}{BSc Thesis}} % Job title
    {University of Florence} % Organisation
    {Florence, Italy} % Location
    {Sept 2020 - Mar 2021} % Date(s)
    {
      \begin{cvitems} % Description(s) of tasks/responsibilities
        \item {Supervisor: Francesca Romana Nardi (University of Florence)}
        \item {Title (translated): Big Deviations for Markov Chains}
        % \item {The thesis focused on presenting a theorem describing the asymptotic behavior of Markov chains for studying rare events within the context of large deviations theory.}
      \end{cvitems}
    }

%---------------------------------------------------------


\end{cventries}
